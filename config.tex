\RequirePackage[english, bahasa]{babel} 	% paket bahasa
\RequirePackage{etoolbox}				% untuk daftar isi
\RequirePackage{setspace}
\RequirePackage{indentfirst}
\RequirePackage[T1]{fontenc}
\RequirePackage{times}
\RequirePackage{graphicx,latexsym}

\usepackage{tocloft} 					% untuk daftar isi
\usepackage{amsfonts, amsmath, amssymb, array}
\usepackage[bookmarks]{hyperref}
\usepackage{bookmark, booktabs, bbding}
\usepackage{calc}						% untuk chapter dan daftar isi
\usepackage{ulem}

\usepackage[thinlines]{easytable}
\usepackage{wrapfig}

\usepackage[inline]{enumitem}
\usepackage{fontawesome}
\usepackage{float}						% ugmtesis
\usepackage{lscape, ltablex}
\usepackage{microtype}
\usepackage{multirow}					% ugmtesis
\usepackage{natbib}
\usepackage{paralist, pdflscape, pifont}
\usepackage{placeins}					% ugmtesis
\usepackage{subfig}
\usepackage{tabularx, titlesec}
\usepackage{wasysym}
\usepackage{xpatch}						% remove gap between chapter in list of figure

%untuk tabel
\usepackage[utf8]{inputenc}
\usepackage[table]{xcolor}
\usepackage{longtable}					% ugmtesis

\usepackage{csquotes}					% ugmtesis

% untuk bingkai
\usepackage{tikz}
\usetikzlibrary{fit}
\usetikzlibrary{calc}

\usepackage{multido}

\usepackage{verbatim}					%multiple coment

\usepackage{geometry}

\usepackage{pdfpages}	% include pdf file

\hypersetup{
	colorlinks=true,
	linkcolor=black,
	citecolor=black,
	filecolor=magenta,
	urlcolor=blue,
}
\bookmarksetup{
	numbered, 
	open,
}
\urlstyle{same}

% menghilangkan space antar item di enumarate
\let\itemize\compactitem
\let\enditemize\endcompactitem
\let\enumerate\compactenum
\let\endenumerate\endcompactenum
\let\description\compactdesc
\let\enddescription\endcompactdesc
\pltopsep=1pt
\plitemsep=1pt
\plparsep=1pt
%-----------------------------------

% membuat ... pada daftar isi untuk section
\makeatletter
\patchcmd{\l@section}
{\hfil}
{\leaders\hbox{\normalfont$\m@th\mkern \@dotsep mu\hbox{.}\mkern \@dotsep mu$}\hfill}

%------------------------------------------------------------]
%Define thesis's inputs
%------------------------------------------------------------
\gdef\@university{Universitas Gadjah Mada}
\gdef\@faculty{Fakultas Matematika dan Ilmu Pengetahuan Alam}
\gdef\@dept{Departemen Ilmu Komputer dan Elektronika}
\gdef\@program{S2 Ilmu Komputer}
\gdef\@jurusan{Ilmu-Ilmu MIPA}
\gdef\@degree{Master of Science}
\gdef\@city{Yogyakarta}
\gdef\@adminS2ILKOM{Sudarini}
\gdef\@headprogram{Dr. Tri Kuntoro Priyambodo, M.Sc.}
\gdef\@headprogramnip{NIP. 19591121 1988 03 1 001}
\gdef\@headdept{Prof. Dr. Triyono, S.U.}
\gdef\@headdeptnip{NIP. 19600921 1986 02 1 001}

\newcommand{\titleind}[1]{\gdef\@titleind{#1}}
\newcommand{\@titleind}{}
\newcommand{\subtitleind}[1]{\gdef\@subtitleind{#1}}
\newcommand{\@subtitleind}{}
\newcommand{\titleeng}[1]{\gdef\@titleeng{#1}}
\newcommand{\@titleeng}{}
\newcommand{\subtitleeng}[1]{\gdef\@subtitleeng{#1}}
\newcommand{\@subtitleeng}{}

\newcommand{\fullname}[1]{\gdef\@fullname{#1}}
\newcommand{\@fullname}{}
\newcommand{\idnum}[1]{\gdef\@idnum{#1}}
\newcommand{\@idnum}{}
\newcommand{\minat}[1]{\gdef\@minat{#1}}
\newcommand{\@minat}{}
\newcommand{\yearsubmit}[1]{\gdef\@yearsubmit{#1}}
\newcommand{\@yearsubmit}{}

\newcommand{\examdate}[1]{\gdef\@examdate{#1}}
\newcommand{\@examdate}{\number\day~\ifcase\month\or
	Januari\or Pebruari\or Maret\or April\or Mei\or Juni\or
	Juli\or Agustus\or September\or Oktober\or Nopember\or Desember\fi
	\space \number\year}
\newcommand{\examday}[1]{\gdef\@examday{#1}}
\newcommand{\@examday}{}
\newcommand{\examtime}[1]{\gdef\@examtime{#1}}
\newcommand{\@examtime}{}
\newcommand{\dateapprove}[1]{\gdef\@dateapprove{#1}}
\newcommand{\@dateapprove}{}

\newcommand{\firstsupervisor}[1]{\gdef\@firstsupervisor{#1}}
\newcommand{\@firstsupervisor}{}
\newcommand{\firstsupervisornip}[1]{\gdef\@firstsupervisornip{#1}}
\newcommand{\@firstsupervisornip}{}
\newcommand{\secondsupervisor}[1]{\gdef\@secondsupervisor{#1}}
\newcommand{\@secondsupervisor}{}
\newcommand{\secondsupervisornip}[1]{\gdef\@secondsupervisornip{#1}}
\newcommand{\@secondsupervisornip}{}

\newcommand{\firstexaminer}[1]{\gdef\@firstexaminer{#1}}%
\newcommand{\@firstexaminer}{}
\newcommand{\firstexaminernip}[1]{\gdef\@firstexaminernip{#1}}
\newcommand{\@firstexaminernip}{}
\newcommand{\secondexaminer}[1]{\gdef\@secondexaminer{#1}}%
\newcommand{\@secondexaminer}{}
\newcommand{\secondexaminernip}[1]{\gdef\@secondexaminernip{#1}}
\newcommand{\@secondexaminernip}{}
\newcommand{\thirdexaminer}[1]{\gdef\@thirdexaminer{#1}}%
\newcommand{\@thirdexaminer}{}
\newcommand{\thirdexaminernip}[1]{\gdef\@thirdexaminernip{#1}}
\newcommand{\@thirdexaminernip}{}

\newcommand{\noRumah}[1]{\gdef\@noRumah{#1}}
\newcommand{\@noRumah}{}
\newcommand{\noKantor}[1]{\gdef\@noKantor{#1}}
\newcommand{\@noKantor}{}
\newcommand{\noHP}[1]{\gdef\@noHP{#1}}
\newcommand{\@noHP}{}
\newcommand{\email}[1]{\gdef\@email{#1}}
\newcommand{\@email}{}
\newcommand{\angkatan}[1]{\gdef\@angkatan{#1}}
\newcommand{\@angkatan}{}

\newcommand{\ipkasal}[1]{\gdef\@ipkasal{#1}}
\newcommand{\@ipkasal}{}
\newcommand{\tahunlulusasal}[1]{\gdef\@tahunlulusasal{#1}}
\newcommand{\@tahunlulusasal}{}
\newcommand{\prodiasal}[1]{\gdef\@prodiasal{#1}}
\newcommand{\@prodiasal}{}
\newcommand{\jurusanasal}[1]{\gdef\@jurusanasal{#1}}
\newcommand{\@jurusanasal}{}
\newcommand{\facultyasal}[1]{\gdef\@facultyasal{#1}}
\newcommand{\@facultyasal}{}
\newcommand{\universityasal}[1]{\gdef\@universityasal{#1}}
\newcommand{\@universityasal}{}

\newcommand{\alamat}[1]{\gdef\@alamat{#1}}
\newcommand{\@alamat}{}
\newcommand{\tmplahir}[1]{\gdef\@tmplahir{#1}}
\newcommand{\@tmplahir}{}
\newcommand{\tgllahir}[1]{\gdef\@tgllahir{#1}}
\newcommand{\@tgllahir}{}
\newcommand{\sksbatal}[1]{\gdef\@sksbatal{#1}}
\newcommand{\@sksbatal}{}
\newcommand{\sksbatalsebelum}[1]{\gdef\@sksbatalsebelum{#1}}
\newcommand{\@sksbatalsebelum}{}
\newcommand{\sksbatalsesudah}[1]{\gdef\@sksbatalsesudah{#1}}
\newcommand{\@sksbatalsesudah}{}
\newcommand{\sksD}[1]{\gdef\@sksD{#1}}
\newcommand{\@sksD}{}
\newcommand{\matakuliahbatal}[1]{\gdef\@matakuliahbatal{#1}}
\newcommand{\@matakuliahbatal}{}
\newcommand{\kodebatal}[1]{\gdef\@kodebatal{#1}}
\newcommand{\@kodebatal}{}
\newcommand{\nilaibatal}[1]{\gdef\@nilaibatal{#1}}
\newcommand{\@nilaibatal}{}
\newcommand{\agama}[1]{\gdef\@agama{#1}}
\newcommand{\@agama}{}

\newcommand{\mastermasuk}[1]{\gdef\@mastermasuk{#1}}
\newcommand{\@mastermasuk}{}
\newcommand{\masterlulus}[1]{\gdef\@masterlulus{#1}}
\newcommand{\@masterlulus}{}
\newcommand{\thakademikmasuk}[1]{\gdef\@thakademikmasuk{#1}}
\newcommand{\@thakademikmasuk}{}
\newcommand{\thakademiklulus}[1]{\gdef\@thakademiklulus{#1}}
\newcommand{\@thakademiklulus}{}
\newcommand{\semesterlulus}[1]{\gdef\@semesterlulus{#1}}
\newcommand{\@semesterlulus}{}
\newcommand{\semestermasuk}[1]{\gdef\@semestermasuk{#1}}
\newcommand{\@semestermasuk}{}

\newcommand{\nilaiTestEnglish}[1]{\gdef\@nilaiTestEnglish{#1}}
\newcommand{\@nilaiTestEnglish}{}
\newcommand{\tahunTest}[1]{\gdef\@tahunTest{#1}}
\newcommand{\@tahunTest}{}
\newcommand{\job}[1]{\gdef\@job{#1}}
\newcommand{\@job}{}
\newcommand{\instansi}[1]{\gdef\@instansi{#1}}
\newcommand{\@instansi}{}
\newcommand{\instansialamat}[1]{\gdef\@instansialamat{#1}}
\newcommand{\@instansialamat}{}
\newcommand{\instansitelp}[1]{\gdef\@instansitelp{#1}}
\newcommand{\@instansitelp}{}

\usepackage{enumitem}
\setlist{parsep=0pt,listparindent=\parindent,
	itemindent=1.5em,labelsep=2em,wide=\parindent
}

\patchcmd{\@chapter}{\addtocontents{lof}{\protect\addvspace{10\p@}}}{}{}{}
\patchcmd{\@chapter}{\addtocontents{lot}{\protect\addvspace{10\p@}}}{}{}{}

%untuk bingkai
\newcommand\HRule{\rule{\textwidth}{1pt}}

\newcommand{\head}[1]{\textnormal{\textbf{#1}}}
\renewcommand{\labelenumii}{\arabic{enumii}.}
\newcounter{colnumbers}
\newcommand{\colnumber}{\stepcounter{colnumbers}\arabic{colnumbers}}
\newcounter{rownumbers}
\newcommand{\rownumber}{\stepcounter{rownumbers}\arabic{rownumbers}}

\newcommand{\soutthick}[1]{%
    \renewcommand{\ULthickness}{2.4pt}%
       \sout{#1}%
    \renewcommand{\ULthickness}{.4pt}% Resetting to ulem default
}

\newcommand{\Pointilles}[2][3]{%\par\nobreak
  \noindent\rule{0pt}{\baselineskip}% Provides a larger gap between the preceding paragraph and the dots
  \multido{}{#2}{\noindent\makebox[0.8\linewidth]{\rule{0pt}{#1\baselineskip}\dotfill}\endgraf}% ... dotted lines ...
  % \bigskip Gap between dots and next paragraph
}

%circled string
\newcommand*\circled[1]{\tikz[baseline=(char.base)]{
            \node[shape=circle,draw,ultra thick,inner sep=2pt] (char) {#1};}}
